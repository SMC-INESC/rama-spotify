% -----------------------------------------------
% Template for ISMIR 2014
% (based on earlier ISMIR templates)
% -----------------------------------------------

\documentclass{article}

\usepackage{ismir2014,amsmath,cite,url,graphicx}
\usepackage[utf8]{inputenc}

\title{Spotify-ed: Music Recommendation and Discovery in Spotify}

% Single address
% To use with only one author or several with the same address
% ---------------
%\oneauthor
% {José Bateira}
% {Affiliations should be omitted for double-blind reviewing}

% Two addresses
% --------------
%\twoauthors
%  {José Bateira} {School \\ Department}
%  {Second author} {Company \\ Address}

% Three addresses
% --------------
\threeauthors
  {José Bateira} {Porto, Portugal \\ {\tt ei10133@fe.up.pt}}
  {Second author} {\bf Retain these fake authors in\\\bf submission to preserve the formatting}
  {Third author} {Affiliation3 \\ {\tt author3@ismir.edu}}

% Four addresses
% --------------
%\fourauthors
%  {José Bateira} {Affiliation1 \\ {\tt author1@ismir.edu}}
%  {Second author}{Affiliation2 \\ {\tt author2@ismir.edu}}
%  {Third author} {Affiliation3 \\ {\tt author3@ismir.edu}}
%  {Fourth author} {Affiliation4 \\ {\tt author4@ismir.edu}}

\begin{document}
%
\maketitle
%
\begin{abstract}
  This demo presents the Spotify App\cite{spotifyapps} of RAMA \cite{ramaapp} that aims to improve the way music recommendations are show in Spotify\footnote{\url{http://spotify.com}}.
  RAMA uses a graphical representation for the relation between music artists in the form of a graph. The nodes are music artists and the edges represent connections between them.
  The information used to construct the graph is provided by Spotify's API.
  The application allows the user to edit the visualization parameters (to obtain more detailed graphs), edit the graph by adding and removing node artists (for users that want to meticulously edit the graph) and visualize the tags/genres that describe an artist in the graph.


\end{abstract}

  \section{Music Discovery Tools in Spotify} % (fold)
  \label{sec:spotify_tools}
  
    Spotify provides a set of tools for the users to discover new music.
    Some are tailored for the user, while others show what is popular at the moment.

    On "Browse" mode, users can see the popular playlists, genres and moods, new releases and news.
    On "Activity" mode, the user can see a feed of the activity from the users that he/she follows, for example, listens and shares.
    The "Discovery" mode is the personalized recommendation system that is powered by the user's followers activity, his/her listening history and more (Examples: "You listened to X. Check out Y.").
    There is also the "Radio" mode that allows the user to start a radio from a single track.
    The user can then, on each track, approve or disapprove the recommendation so as to improve the quality of the recommendations provided.
    The other possibilities for the user to discover new music rely on the third-party Spotify apps that other services developed. 
    For example, Last.fm\footnote{\url{http://last.fm}} developed their application that uses its recommendation system. 
    This way, the Spotify user can have the best of both worlds from Spotify and Last.fm.

    All of these approaches have on thing in common: the way the recommendations are presented to the user are in the form of lists and/or grids.


  % section spotify_tools (end)

  \section{RAMA in Spotify} % (fold)
  \label{sec:rama}
  
  RAMA breaks this pattern by introducing a visual representation for recommendations.0

  % section rama (end)

  \section{Main Features} % (fold)
  \label{sec:features}
  


  % section features (end)

% \begin{thebibliography}{citations}

% \bibitem {Author:00}
% E. Author:
% ``The Title of the Conference Paper,''
% {\it Proceedings of the International Symposium
% on Music Information Retrieval}, pp.~000--111, 2000.

% \bibitem{Someone:10}
% A. Someone, B. Someone, and C. Someone:
% ``The Title of the Journal Paper,''
% {\it Journal of New Music Research},
% Vol.~A, No.~B, pp.~111--222, 2010.

% \bibitem{Someone:04} X. Someone and Y. Someone: {\it Title of the Book},
%     Editorial Acme, Porto, 2012.

% \end{thebibliography}

\bibliography{../report/bib/myrefs}
\bibliographystyle{unstrnat}

\end{document}
