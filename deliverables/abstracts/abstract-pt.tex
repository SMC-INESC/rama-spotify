%-----------------------------------------------
% Template para criação de resumos de projectos/dissertação
% jlopes AT fe.up.pt,   Fri Jul  3 11:08:59 2009
%
% actualizado por
% jlageb AT gmail.com
%-----------------------------------------------

\documentclass[twocolumn]{article}

\usepackage[portuguese]{babel}
\usepackage{graphicx}           % images .png or .pdf w/ pdflatex OR .eps w/ latex
\usepackage[utf8]{inputenc}     % 8 bits using UTF-8
\usepackage{url}                % URLs
\usepackage{float}              % improve figures & tables floating
% \usepackage[tableposition=top]{caption} % captions
\usepackage{indentfirst}        % portuguese standard for paragraphs

% shift the title up a bit
\usepackage{titling}
\setlength{\droptitle}{-10em}

%% headers & footers
\pagestyle{empty}

% title begin
\title{
  \huge
  \textbf{
    Spotify-ed: Music Recommendation and Discovery in Spotify
  }
}
\author{
  \large{\emph{\textbf{José Lage Bateira}}} \\ \\
  Supervisores: \emph{Fabien Gouyon e Matthew Davies}  \\
  no \emph{INESC Porto}
}
\date{23 de Julho, 2014}

\begin{document}

\maketitle

%no page number
\thispagestyle{empty}

\section{Motivação}\label{sec:motiva}

  Bem longe vão os tempos, antes da Internet, em que ouvir e descobrir música nova era um desafio por si só.
  Agora, com alguns cliques, temos acesso a um catálogo de música tão grande, que o nosso cérebro não consegue processar.

  Existem dezenas de serviços online que oferecem isso mesmo.
  Alguns especializam-se na criação/geração de playlists (que funcionam como rádios), outros em expandir o catálogo de música e outros focam-se mais na sugestão e recomendação de artistas/álbuns/músicas personalizada para os utilizadores.
  Estes últimos, apresentam as sugestões de conteúdo ao utilizador de uma forma rudimentar como listas ou em grelha.

  No entanto, listas ou grelhas não fornecem ao utilizador qualquer tipo de informação adicional sobre a relação entre os artistas nem justificam a sua semelhança \cite{Lamere2008}.
  Até fazem parecer que não existe nenhuma relação/ligação entre os artistas recomendados, o que não é verdade.

  Essas relações existem e podem ser representadas como uma rede de artistas interligados num grafo, onde cada nó é um artista de música, e cada ligação entre nós representa uma ligação forte de parecença entre os artistas.
  Este é o conceito que o RAMA (Relational Artist MAps), projeto desenvolvido no INESC Porto, usa \cite{Costa2008} \cite{Sarmento2009} \cite{Costa2009} \cite{Gouyon2011}.

\section{Objetivos}\label{sec:goals}

  A partir de uma pesquisa de um artista de música, o RAMA cria e desenha um grafo que ajuda o utilizador a explorar música que lhe possa interessar de uma forma muito mais natural e informativa.

  No entanto, quando um utilizador pretende ouvir uma música de um artista, é usado \emph{stream} do Youtube. 
  Apesar de este oferecer um catálogo alargado de música, o mesmo não é indicado para esta funcionalidade pois não fornece uma API nativamente orientada a música, nem a qualidade de som do \emph{stream} é adequada.

  A experiência musical do utilizador do RAMA poderá melhorar consideravelmente ao colmatar esta falha.
  Existe por isso uma necessidade de substituir o Youtube por outro serviço mais orientado a \emph{streaming} de música de qualidade.
  O Spotify é um deles. Fornece API orientada a música, e o \emph{streaming} é de qualidade adequada para este tipo de funcionalidade.

  De que formas é que se pode integrar o RAMA e o Spotify?

  A proposta inicial tinha como objectivo desenvolver, pelo menos, um dos seguintes módulos:

  \begin{enumerate}
    \item Integrar o serviço de \emph{Streaming} to Spotify no RAMA
    \item Integrar informação de um utilizador Spotify no RAMA
    \item Melhorar funcionalidades e design do RAMA
    \item Integrar o conceito do RAMA numa Aplicação Spotify.
    \item Integrar geração de \emph{playlists} numa Aplicação Spotify.
    \item Integrar algumas das funcionalidades anteriores numa Aplicação Móvel.
  \end{enumerate}

  A escolha final foi desenvolver uma aplicação (como \emph{plugin}) para o Spotify.
  Será que um utilizador Spotify ao descobrir música nova de uma forma mais gráfica terá uma experiência de utilizador mais rica e natural do que o modo de descoberta \emph{standard} do Spotify (em grelha)?

\section{Descrição do Trabalho}\label{sec:work}

  As fases de trabalho completadas, são:

  \begin{description}
    \item[Estado da Arte] \hfill \\
      Investigação inicial do estado atual de arte.
    Inclui os serviços que oferecem ao utilizador uma plataforma completa com \emph{stream} de um catálogo de artistas grande e ferramentas de descoberta de música.
    No entanto, apenas se irá entrar em detalhe nos serviços que usam ferramentas visuais.

    \item[Contextualização] \hfill \\
      Análise detalhada do ambiente que o Spotify fornece, tanto de um perspectiva de utilizador (que aplicações este tem disponível) como de uma perspectiva de \emph{developer} (\emph{APIs} disponíveis).

    \item[Implementação e Validação] \hfill \\
      Definição e implementação: das funcionalidades principais do protótipo a desenvolver; dos processos de desenvolvimento e de validação junto dos utilizadores.

    \item[Discussão e Trabalho Futuro] \hfill \\
      Discussão dos resultados e definição de trabalho futuro a ser implementado no protótipo.
  \end{description}

  A principais funcionalidades desenvolvidas no protótipo são:

  \begin{itemize}
    \item Visualização das relações entre os artistas de música usando uma ferramenta visual;
    \item Edição dos parâmetros de visualização;
    \item Edição do grafo eliminando e acrescentando nós;
    \item Visualização de \emph{tags}/géneros (que descrevam um artista) na representação gráfica.
  \end{itemize}


\section{Conclusões}\label{sec:conclui}

  Aplicando o conceito do RAMA no protótipo desenvolvido, a experiência de utilizador ao descobrir música nova melhorou consideravelmente.
  Todos os utilizadores que testaram a aplicação gostaram da experiência visual e a maioria respondeu positivamente quando lhes foi perguntado se usariam a aplicação para descobrir música nova.

\bibliographystyle{unsrt}
\bibliography{../report/bib/myrefs}

\end{document}
