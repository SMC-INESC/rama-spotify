%-----------------------------------------------
% Template para criação de resumos de projectos/dissertação
% jlopes AT fe.up.pt,   Fri Jul  3 11:08:59 2009
%
% actualizado
% jlageb AT gmail.com
%-----------------------------------------------

\documentclass[twocolumn]{article}

\usepackage[english]{babel}
\usepackage{graphicx}
\usepackage[utf8]{inputenc}
\usepackage{url}
\usepackage{float}
% \usepackage[tableposition=top]{caption} % captions

% shift the title up a bit
\usepackage{titling}
\setlength{\droptitle}{-10em}

%% headers & footers
\pagestyle{empty}

% title begin
\title{
  \huge
  \textbf{
    Spotify-ed: Music Recommendation and Discovery in Spotify
  }
}
\author{
  \large{\emph{\textbf{José Lage Bateira}}} \\ \\
  Supervisors: \emph{Fabien Gouyon and Matthew Davies}  \\
  at \emph{INESC TEC}
}
\date{23\textsuperscript{rd} of June, 2014}

\begin{document}

\maketitle

%no page number
\thispagestyle{empty}

\section{Motivations}
\label{sec:motivations}

  Not so long ago, before the Internet boom, listening or discovering new music was a challenge on its own.
  Now, with a few clicks, users can have on their hands such a vast music catalogue that a human mind is not able to compute.

  There is an uncountable number of music streaming services that offer exactly that\footnote{Although some of them require the users to subscribe to a monthly fee, for example, in order to fully use the service, or remove the advertisements.}.
  These services are, mostly, web based, although some offer desktop applications.
  They allow the users to play music, save their collection, create playlists and much more.
  Most of these services also have social components that allow the users to share what they're listening to with their friends, as well as playlists.

  There is always something that makes a music streaming service different from the others.
  Some services focus on creation and/or generation of playlists (8tracks~\cite{8tracks}), others try to expand their music catalogue even further (Spotify~\cite{spotify}, Rdio~\cite{rdio}), while others focus more on personalized music recommendations (Pandora~\cite{pandora}).
  The latter ones, present their music recommendations to the user with a list or a grid of music artists, for example.
  However, lists do not provide the user enough information about the relation between the results \cite{Lamere2008}.

  A possible solution to this problem is to represent the artist's similarities as a network of interconnected artists in a graph, where a node is a music artist, and each edge between them represents a connection.
  This is the concept that RAMA (Relational Artist MAps)\footnote{RAMA: \url{http://rama.inescporto.pt}}, a project developed at INESC Porto, uses \cite{Costa2008} \cite{Sarmento2009} \cite{Costa2009} \cite{Gouyon2011}.
  

\section{Goals}
\label{sec:goals}

  From a single search, the original RAMA system draws a graph that helps the user to explore new music that might caught his/her interest in a much more natural way.
  Nonetheless, when a user wants to listen to an artist's music, Youtube's stream is used.
  Although one can find a large catalogue of music in Youtube, this service is not music oriented and the sound quality is not adequate for a music streaming service.
  Youtube's stream needs to be replaced.
  From the available services that provide a vast music catalogue, Spotify\footnote{\url{http://spotify.com}} provides a good quality stream and a good developer support for creating Spotify powered Applications.
  But how can RAMA and Spotify be integrated?
  
  The initial proposal was to develop a software module that implements, at least, one of the following features:

  \begin{enumerate}
    \item \label{intro:obj1} Integrate Spotify's music stream into RAMA's website
    \item \label{intro:obj2} Integrate information from the Spotify user into RAMA
    \item \label{intro:obj3} Improve RAMA's features and design
    \item \label{intro:obj4} Integrate the RAMA concept into a Spotify Application
    \item \label{intro:obj5} Integrate RAMA's playlist generation into a Spotify Application
    \item \label{intro:obj6} Integrate some of the above mentioned modules into a Mobile Application
  \end{enumerate}

  In the end, the final proposal is a Spotify Application~\cite{spotifyapps} (module~\ref{intro:obj4}) that works like a plugin to the Spotify's Desktop Client, i.e., it should add something to Spotify.
  This is a very appealing solution: Spotify Users will have the chance to continue using Spotify as they would normally do, but with an extra help to discover new music by using RAMA's application \emph{inside} Spotify. 
  This method works on the assumption that Spotify's music discovery mode can be improved using a visual tool like RAMA.

\section{Work Description}
\label{sec:work}
  
  The completed work phases are:

  \begin{description}
    \item[State of the Art] \hfill \\
      Initial research on the current state of the art. 
      This includes the services that provide a platform for users to listen and discover new music. 
      Focus will be given to the ones that use visual tools.
    \item[Contextualization] \hfill \\
      Detailed analysis of the Spotify environment from the users perspective (applications available, e.g.) and from the developers perspective (available APIs, e.g.) in order to give a much more insightful view when determining the feasibility of the prototype's requirements.
    \item[Implementation and Validation] \hfill \\
      Definition and implementation of: the prototype's main features/requirements; the development processes and the user validation processes.
    \item[Discussion and Future Work] \hfill \\
      Discussion of the results and definition of future work to be done in the prototype (improvements, features, bugs, etc).
  \end{description}

  And the main features of the implemented prototype are as follows:

  \begin{itemize}
    \item Representation of relations between artists by means of a visual tool;
    \item Edition of the visualization using several parameters;
    \item Edition of the graph by allowing to remove and add new nodes;
    \item Visualize tags/genres (that describe an artist) in the graph representation.
  \end{itemize}


\section{Conclusions}
\label{sec:conclusions}

  By using RAMA's concept applied in the developed prototype, the user experience when discovering new music as been greatly increased.
  All of the beta-testers liked the visual experience and the majority responded positively about using the application in a regular basis to discover new music.

\bibliographystyle{../report/bib/unsrt-pt}
\bibliography{../report/bib/myrefs}

\end{document}
