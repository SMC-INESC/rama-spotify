%!TEX root = ../mieic.tex

\chapter{Revisão Bibliográfica} \label{chap:chap2}

\section*{}

\section{Introdução}

Esta dissertação foca-se mais na forma como se apresenta o conteúdo que se pretende recomendar ao utilizador, e não qual o conteúdo que é sugerido (não obstante da sua importância obviamente).
No entanto, é quase impossível, no estudo do estado da arte, não se refirir outros projetos que se focam também no conteúdo.

Regra geral, os projetos que de seguida serão analisados, utilizam bases de dados externas, como o last.fm, para obter metadata que, convenientemente, também oferecem um tipo de recomendação de música com base numa pesquisa inicial.


\section{Projetos Relacionados} % (fold)
\label{sec:projetos_relacionados}


\subsection{liveplasma.com} % (fold)
\label{sub:projeto_1}



% subsection projeto_1 (end)

\subsection{audiomap.tuneglue.net} % (fold)
\label{sub:projeto_2}


% subsection projeto_2 (end)

\subsection{Discovr.info} % (fold)
\label{sub:projeto_3}


% subsection projeto_3 (end)

% section projetos_relacionados (end)


\section{Resumo ou Conclusões}

No final do capítulo deverá ser apresentado um resumo com as 
principais conclusões que se podem tirar. 
