%!TEX root = ../mieic.tex

\chapter{Projeto}
\label{chap:chap3}

\section*{}

O principal objetivo desta dissertação, como foi referido no capítulo \ref{chap:intro}, é desenvolver um (ou mais) módulo(s) de software que contribuam para uma melhoria na descoberta e recomendação de música num ambiente integrado entre o RAMA e o Spotify, por forma a tirar partido da representação gráfica do grafo de artistas de música do RAMA e da qualidade do serviço de \emph{Streaming} de música do Spotify.

Para tal, a proposta inicial desta dissertação consiste em desenvolver, no mínimo, um módulo que implemente uma das seguintes funcionalidades:

\begin{enumerate}
  \item \label{item:obj1} Integrar o serviço de \emph{streaming} de música do Spotify \textbf{no RAMA}
  \item \label{item:obj2} Integrar informação de um utilizador Spotify \textbf{no RAMA}
  \item \label{item:obj3} Melhorar design e funcionalidades \textbf{do RAMA}
  \item \label{item:obj4} Integrar a visualização de grafos de artistas de música \textbf{numa Aplicação Spotify}
  \item \label{item:obj5} Integrar o módulo de criação de \emph{playlists} do RAMA \textbf{numa Aplicação Spotify}
  \item \label{item:obj6} Integrar alguns dos módulos acima referidos \textbf{numa aplicação móvel}
\end{enumerate}

As três primeiras funcionalidades (\ref{item:obj1}, \ref{item:obj2} e \ref{item:obj3}) focam-se em melhorar o serviço do RAMA, usando API's do Spotify, ou seja, integrar o Spotify dentro do RAMA.
Por outro lado, as funcionalidades \ref{item:obj4} e \ref{item:obj5} têm como objetivo integrar o RAMA dentro do Spotify, através de uma Aplicação Spotify, que funciona como plugin do programa principal do Spotify.
A última funcionalidade (\ref{item:obj6}) teria de implementar algumas das anteriores num Sistema Operativo Móvel (Android, iOS ou Windows Phone).

Este capítulo procura analisar todas as condicionantes que afetam a escolha  dos módulos a desenvolver, e em que ambientes estes se encaixam melhor (Aplicação Spotify, aplicação móvel ou RAMA).

Inicialmente será explorado o ambiente de desenvolvimento que o Spotify disponibiliza, ou seja, que tecnologias tem disponíveis para \emph{developers}.
De seguida serão analisadas quais dessas tecnologias assentam melhor em cada um dos módulos propostos a desenvolver, através de experimentações feitas\footnote{E quando necessário, será descrito um possível esquema de arquitetura por forma a facilitar a explicação do problema}.


No final, deve ficar claro quais serão os módulos de software a desenvolver, que tecnologias irão ser usadas e qual o esquema geral da sua arquitetura.
O produto final deve de ir ao encontro do objetivo de contribuir para uma melhoria na descoberta e recomendação de música num ambiente relacionado com o RAMA.


\section{Spotify} % (fold)
\label{sec:spotify}

  O Spotify é um serviço de \emph{streaming} de música que...

  \subsection{Ferramentas de Desenvolvimento} % (fold)
  \label{sub:ferramentas_de_desenvolvimento}
  
    No momento de escrita deste relatório, o Spotify tem disponível um conjunto de ferramentas\footnote{http://developer.spotify.com/technologies} para desenvolver módulos de software que podem estar embebidos nas mais diversas aplicações (\emph{third-party applications}) ou então dentro do \emph{Spotify Desktop Client}.

    Existem quatro ferramentas de desenvolvimento, cada uma delas com o seu propósito e utilidade.


    \subsubsection{Spotify Apps} % (fold)
    \label{ssub:spotify_apps}
      É usada para desenvolver Aplicações Spotify\footnote{https://developer.spotify.com/technologies/apps} que são usadas pelos utilizadores Spotify dentro do \emph{Spotify Desktop Client}.
      
      Para o seu desenvolvimento são disponibilizadas duas \emph{frameworks}: \emph{API Framework}\footnote{https://developer.spotify.com/docs/apps/api/1.0/} e \emph{Views Framework}\footnote{https://developer.spotify.com/docs/apps/views/1.0/}.
      A primeira fornece uma interface para recolher metadados de artistas, álbuns e músicas e controlar o reprodutor de música.
      A segundar fornece componentes de design como botões, listas, abas, entre outros, para o desenvolvimento da aplicação.

      Para desenvolver os módulos \ref{item:obj4} e \ref{item:obj5}, será necessário

    % subsubsection spotify_apps (end)


    \subsubsection{Spotify Widgets} % (fold)
    \label{ssub:spotify_widgets}
      Spotify Widgets\footnote{https://developer.spotify.com/technologies/widgets}


    % subsubsection spotify_widgets (end)

    \subsubsection{Libspotify SDK} % (fold)
    \label{ssub:libspotify_sdk}
    
      Libspotify SDK\footnote{https://developer.spotify.com/technologies/libspotify}

    % subsubsection libspotify_sdk (end)


    \subsubsection{Metadata API} % (fold)
    \label{ssub:metadata_api}
    
      Metadata API\footnote{https://developer.spotify.com/technologies/web-api}

    % subsubsection metadata_api (end)

  % subsection ferramentas_de_desenvolvimento (end)

% section spotify (end)

\section{Tecnologias} % (fold)
\label{sec:tecnologias}

% section tecnologias (end)

\section{Arquitetura} % (fold)
\label{sec:arquitetura}


% section arquitetura (end)

\section{Experimentação Feita} % (fold)
\label{sec:experimentacao}

% section experimentacao (end)

\section{Resumo e Conclusões}


