%!TEX root = ../mieic.tex

\chapter{Projeto}\label{chap:chap3}

\section*{}

Este capítulo deve começar por fazer uma apresentação detalhada do
problema a resolver\footnote{Na introdução a apresentação do
  problema foi breve.} podendo mesmo, caso se justifique,
constituir-se um capítulo com essa finalidade.

Deve depois dedicar-se à apresentação da solução sem detalhes de
implementação. 
Dependendo do trabalho, pode ser uma descrição mais teórica, mais
``arquitetural'', etc.

\section{Spotify} % (fold)
\label{sec:spotify}

\lipsum

% section spotify (end)

\section{Tecnologias} % (fold)
\label{sec:tecnologias}

% section tecnologias (end)

\section{Arquitetura} % (fold)
\label{sec:arquitetura}

\lipsum

% section arquitetura (end)

\section{Experimentação Feita} % (fold)
\label{sec:experimentacao}

% section experimentacao (end)

\section{Resumo e Conclusões}

Resumir e apresentar as conclusões que se podem tirar no fim deste
capítulo.
