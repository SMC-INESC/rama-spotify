%!TEX root = ../mieic.tex

\chapter{Introdução} \label{chap:intro}


\section*{}

\section{Contexto e Enquadramento} \label{sec:context}

Bem longe vão os tempos, antes da Internet, em que ouvir e descobrir música nova era um desafio por si só.
Agora, com alguns cliques, temos acesso a um catálogo de música tão grande, que o nosso cérebro não consegue processar.

Existem dezenas de serviços online que oferecem isso mesmo.
Alguns especializam-se na criação/geração de playlists (que funcionam como rádios), outros em expandir o catálogo de música e outros focam-se mais na sugestão e recomendação de artistas/álbuns/músicas personalizada para os utilizadores.
Estes últimos, apresentam as sugestões de conteúdo ao utilizador de uma forma rudimentar como listas ou em grelha.

No entanto, listas ou grelhas não fornecem ao utilizador qualquer tipo de informação adicional sobre a relação entre os artistas nem justificam a sua semelhança \cite{Lamere2008}.
Até fazem parecer que não existe nenhuma relação/ligação entre os artistas recomendados, o que não é verdade.

Essas relações existem e podem ser representadas como uma rede de artistas interligados num grafo, onde cada nó é um artista de música, e cada ligação entre nós representa uma ligação forte de parecença entre os artistas.
Este é o conceito que o RAMA\footnote{http://rama.inescporto.pt}, projeto desenvolvido no INESC Porto\footnote{http://inescporto.pt}, usa. \cite{Costa2008} \cite{Sarmento2009} \cite{Costa2009} \cite{Gouyon2011}



\section{Motivação e Objetivos} \label{sec:goals}


A partir de uma pesquisa de um artista de música, o RAMA cria e desenha um grafo que ajuda o utilizador a explorar música que lhe possa interessar de uma forma muito mais natural e informativa.
A informação que o RAMA usa é retirada do serviço Last.fm\footnote{http://last.fm}.

No entanto, quando um utilizador pretende ouvir uma música de um artista, é usado \emph{stream} do Youtube\footnote{http://youtube.com}. 
Apesar de este oferecer um catálogo alargado de música, o mesmo não é indicado para esta funcionalidade pois não fornece uma API\footnote{Application Programming Interface} nativamente orientada a música, nem a qualidade de som do \emph{stream} é adequada.

A experiência musical do utilizador do RAMA poderá melhorar consideravelmente ao colmatar esta falha.
Existe por isso uma necessidade de substituir o Youtube por outro serviço mais orientado a \emph{streaming} de música de qualidade.
O Spotify\footnote{http://spotify.com} é um deles. Fornece API orientada a música \footnote{https://developer.spotify.com/technologies/web-api}, e o \emph{streaming} é de qualidade adequada para este tipo de funcionalidade.

De que formas é que se pode integrar o RAMA e o Spotify?

Por forma a resolver este problema, foram analisadas várias possibilidades:

\begin{description}
  \item[Spotify Play Button\footnote{https://developer.spotify.com/technologies/widgets/spotify-play-button/}] \hfill \\
    \emph{Widget} do Spotify que pode ser embebida no RAMA.
  \item[Integrar o Perfil de um utilizador Spotify no RAMA] \hfill \\
    Para complementar as recomendações de artistas.
  \item[Aplicação Spotify\footnote{https://developer.spotify.com/technologies/apps/}] \hfill \\
    Serve como \emph{plugin} ao programa do Spotify, estendendo as funcionalidades do Spotify com visualização gráfica de recomendações e construção de playlists.
  \item[Aplicação Móvel] \hfill \\
    Com as funcionalidades acima descritas.
\end{description}

A escolha final foi desenvolver uma aplicação (como \emph{plugin}) para o Spotify.
Será que um utilizador Spotify ao descobrir música nova de uma forma mais gráfica terá uma experiência de utilizador mais rica e natural do que o modo de descoberta \emph{standard} do Spotify (em grelha)?

Esse é o objetivo primordial desta dissertação: Tentar descobrir se utilizadores Spotify terão uma experiência melhorada ao usar a Aplicação Spotify proposta.

No entanto, para avaliar e validar o resultado final, será necessário fazer testes com utilizadores finais para comparar a sua experiência no Spotify com e sem a aplicação desenvolvida.

Desta forma, o desenvolvimento da aplicação será feito de forma iterativa, implementando as funcionalidades que o RAMA oferece, e quando esta estiver ao nível do RAMA, serão realizados testes cuidados com os utilizadores à medida que se melhora a implementação com o seu \emph{feedback}.

\section{Projeto} \label{sec:proj}

A aplicação a desenvolver será uma alternativa ao modo de descoberta do Spotify.
No momento de escrita deste relatório, o modo de descoberta/recomendação de música do Spotify, comparativamente ao do Last.fm, é simples: apresenta recomendações em forma de grelha.

Desta forma, propõe-se uma representação visual em forma de grafo similar à do RAMA.
A aplicação corre dentro do ambiente do Spotify como uma Aplicação Spotify.
As suas principais funcionalidades serão: pesquisa de conteúdo, interação com o grafo e reproduzir música dos resultados da pesquisa.
Estes são os requisitos mínimos que irão ser implementados.
\\
\\
Durante todo o desenvolvimento da aplicação, algumas das ferramentas a ser usadas serão:

\begin{description}
  \item[Spotify Desktop Client] \hfill \\
    O desenvolvimento de aplicações Spotify é feito de forma integrada no programa.
  \item[Webkit Development Tools - webkit.org] \hfill \\
    A aplicação do Spotify foi desenvolvida com Webkit, e por isso, as aplicações Spotify também o são.
  \item[Npmjs - npmjs.org] \hfill \\
    Gestor de pacotes de software e dependências.
  \item[Gruntjs - gruntjs.com] \hfill \\
    Programa de gestão de tarefas automatizadas. Muito útil para testes, compilação e otimização de código.
  \item[Arborjs - arborjs.org] \hfill \\
    Framework de javascript para desenho de grafos. Foi já utilizada no desenvolvimento do RAMA (existe sempre a possibilidade de se usar outra ferramenta substituta caso esta não for adequada).
\end{description}



\section{Estrutura da Dissertação} \label{sec:struct}

Para além da introdução, este relatório contém mais 4 capítulos.

No capítulo~\ref{chap:chap2}, é descrito o estado da arte onde são
apresentados trabalhos relacionados.

No capítulo~\ref{chap:chap3}, é explicado em detalhe em que consiste o projeto, com uma introdução ao ambiente de desenvolvimento de Aplicações Spotify, às tecnologias a serem usadas, à arquitetura e experimentação feita até ao momento.

No capítulo~\ref{chap:chap4}, é descrito com detalhe todo o plano de trabalho que esta dissertação vai seguir. 
Sempre que possível, será explicado com mais detalhe as tarefas mais importantes que estão previstas durante o desenvolvimento deste projeto.

No capítulo~\ref{chap:concl}, são apresentadas conclusões sobre o planeamento do projeto até agora realizado.
