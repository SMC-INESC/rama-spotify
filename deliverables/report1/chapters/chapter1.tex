%!TEX root = ../mieic.tex

\chapter{Introdução} \label{chap:intro}



\section{Contexto/Enquadramento} \label{sec:context}

Esta secção descreve a área em que o trabalho se insere, podendo
referir um eventual projeto de que faz parte e apresentar uma breve
descrição da empresa onde o trabalho decorreu.

\lipsum[1]
\lipsum[1]

\section{Motivação e Objetivos} \label{sec:goals}

Apresenta a motivação e enumera os objetivos do trabalho terminando
com um resumo das metodologias para a prossecução dos objetivos.

\lipsum[1]
\lipsum[1]

\section{Projeto} \label{sec:proj}

Na continuação da secção anterior, e apenas no caso de ser um Projeto
e não uma Dissertação, esta secção apresenta resumidamente o projeto.

\lipsum[1]
\lipsum[1]

\section{Estrutura da Dissertação} \label{sec:struct}

Para além da introdução, esta dissertação contém mais x capítulos.
No capítulo~\ref{chap:sota}, é descrito o estado da arte e são
apresentados trabalhos relacionados.
%\todoline{Complete the document structure.}
No capítulo~\ref{chap:chap3}, é explicado em detalhe em que consiste o projeto
No capítulo~\ref{chap:chap4} praesent sit amet sem. 
No capítulo~\ref{chap:concl}  posuere, ante non tristique
consectetuer, dui elit scelerisque augue, eu vehicula nibh nisi ac
est. 
