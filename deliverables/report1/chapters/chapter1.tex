%!TEX root = ../mieic.tex

\chapter{Introdução} \label{chap:intro}



\section{Contexto/Enquadramento} \label{sec:context}

Bem longe vão os tempos, antes da Internet, em que ouvir e descobrir música nova era um desafio por si só.
Agora, com alguns cliques, temos acesso a um catálogo de música tão grande, que o nosso cérebro não consegue processar.

Existem dezenas de serviços online que oferecem isso mesmo.
Alguns especializam-se na criação/geração de playlists (que funcionam como rádios), outros em expandir o catálogo de música e outros focam-se mais na sugestão e recomendação de artistas/álbuns/músicas personalizada para os utilizadores.
Estes últimos, apresentam as sugestões de conteúdo ao utilizador de uma forma rudimentar como listas ou em grelha.

No entanto, listas ou grelhas não fornecem ao utilizador qualquer tipo de informação adicional sobre a relação entre os artistas nem justificam a sua semelhança \cite{Lamere2008}.
Até fazem parecer que não existe nenhuma relação/ligação entre os artistas recomendados, o que não é verdade.

Essas relações existem e podem ser representadas como uma rede de artistas interligados num grafo, onde cada nó é um artista de música, e cada ligação entre nós representa uma ligação forte de parecença entre os artistas.
Este é o conceito que o RAMA\footnote{http://rama.inescporto.pt}, projeto desenvolvido no INESC Porto\footnote{http://inescporto.pt}, usa. \cite{Costa2008} \cite{Sarmento2009} \cite{Costa2009} \cite{Gouyon2011}

\section{Motivação e Objetivos} \label{sec:goals}


A partir de uma pesquisa de um artista de música, o RAMA cria e desenha um grafo que ajuda o utilizador a explorar música que lhe possa interessar de uma forma muito mais natural e informativa.
A informação que o RAMA usa é retirada do serviço Last.fm\footnote{http://last.fm}.

No entanto, quando um utilizador pretende ouvir uma música de um artista, é usado stream do Youtube\footnote{http://youtube.com}. 
Apesar de este oferecer um catálogo alargado de música, o mesmo não é indicado para esta funcionalidade pois não fornece nenhum tipo de API\footnote{Application Programming Interface} orientada a música, nem a qualidade de som do stream é adequada.

Por forma a resolver este problema, foram analisadas várias possibilidades de integrar um serviço de streaming de música no RAMA, ou vice-versa.
A escolha final foi desenvolver uma aplicação (como plugin) para o Spotify\footnote{http://spotify.com}.
Desta forma, os utilizadores Spotify poderão ter uma experiência mais natural e integrada, usufruindo de todas as funcionalidades que o RAMA oferece, dentro do Spotify (Aplicações Spotify\footnote{http://developer.spotify.com/technologies/apps}).
 

O objetivo final do projeto é oferecer aos utilizadores do Spotify, uma experiência de descoberta de música muito mais agradável, natural e detalhada, recorrendo a uma visualização das relações entre artistas.
No entanto, para avaliar e validar o resultado final, será necessário fazer testes com utilizadores finais para comparar a sua experiência no Spotify com e sem a aplicação desenvolvida.

Desta forma, o desenvolvimento da aplicação será feito de forma iterativa, implementando as funcionalidades que o RAMA oferece, e quando esta estiver ao nível do RAMA, serão realizados testes cuidados com os utilizadores à medida que se melhora a implementação com o seu feedback.
 \\
 \\
Em suma, os objetivos deste projeto são:

\begin{itemize}
  \item Melhorar a forma de apresentação da recomendação de conteúdo dentro do Spotify;
  \item Dar mais poder ao utilizador na descoberta de música nova;
  \item Tornar a experiência de descoberta mais interessante e cativante.
\end{itemize}

\section{Projeto} \label{sec:proj}

A aplicação a desenvolver será uma alternativa ao modo de descoberta do Spotify.
No momento de escrita deste relatório, o modo de d  Description
activityType  module:api/activity~Activity.Type Type of the activity.
item  module:api/models~Loadable  Item, primary object in the activity.
context module:api/models~Loadable  Context, secondary object in the activity.
referrer   {uri: string, name: string, image: string} Referrer of the activity.
message  string Message attached to the activity.
timestamp  Date The timestamp when state was applied.
user  module:api/models~User  User object, owner of the activity.escoberta/recomendação de música do Spotify, comparativamente ao do Last.fm, é simples: apresenta recomendações em forma de grelha.

Desta forma, propõem-se uma representação visual em forma de grafo similar à do RA
  \item [Gruntjs]MA.
A aplicação corre dentro do ambiente do Spotify como uma Aplicação Spotify.
As suas principais funcionalidades serão: pesquisa de conteúdo, interação com o grafo e reproduzir música dos resultados da pesquisa.
Estes são os requisitos mínimos que irão ser implementados.

Durente todo o desenvolvimento da aplicação, algumas das ferramentas a ser usadas são:

\begin{description}
  \item[Spotify Desktop Client] spotify.com
  \item[Webkit Development Tools] webkit.org
  \item[Sublime Text] sublimetext.com
  \item[Bower] bower.io
  \item[Gruntjs] gruntjs.com
  \item[Arborjs] arborjs.org
\end{description}



\section{Estrutura da Dissertação} \label{sec:struct}

\emph{Para além da introdução, esta dissertação contém mais x capítulos.
No capítulo~\ref{chap:sota}, é descrito o estado da arte e são
apresentados trabalhos relacionados.
No capítulo~\ref{chap:chap3}, é explicado em detalhe em que consiste o projeto
No capítulo~\ref{chap:chap4} praesent sit amet sem. 
No capítulo~\ref{chap:concl}  posuere, ante non tristique
consectetuer, dui elit scelerisque augue, eu vehicula nibh nisi ac
est. 
}