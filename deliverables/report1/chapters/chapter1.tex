%!TEX root = ../mieic.tex

\chapter{Introdução} \label{chap:intro}

Bem longe vão os tempos, antes da Internet, em que ouvir e descobrir música nova era um desafio por si só. Agora, com alguns cliques, temos acesso a um catálogo de música tão grande, que o nosso cérebro não consegue processar.

Existem dezenas de serviços online que oferecem isso mesmo.
Alguns especializam-se na criação/geração de playlists (que funcionam como rádios), outros em expandir o catálogo de música e outros focam-se mais na capacidade de sugestão e recomendação de música (ou artistas de música) aos utilizadores.
Estes últimos, apresentam as sugestões de artistas/álbuns/músicas ao utilizador de uma forma rudimentar como listas ou em grelha.

No entanto, listas ou grelhas não fornecem ao utilizador qualquer tipo de informação adicional sobre a relação entre os artistas nem justificam a sua semelhança \cite{Lamere2008}.
Até fazem parecer que não existe nenhuma relação/ligação entre os artistas recomendados, o que não é verdade.

Essas relações existem e podem ser representadas como uma rede de artistas interligados num grafo, onde cada nó é um artista de música, e cada ligação entre nós representa uma ligação forte de parecença entre os artistas.
Este é o conceito que o RAMA\footnote{http://rama.inescporto.pt}, projeto desenvolvido no INESC Porto, usa.

A partir de uma pesquisa de um artista de música, o RAMA cria e desenha um grafo que ajuda o utilizador a explorar música que lhe possa interessar de uma forma muito mais natural e informativa.

\section{Contexto/Enquadramento} \label{sec:context}

\emph{Esta secção descreve a área em que o trabalho se insere, podendo
referir um eventual projeto de que faz parte e apresentar uma breve
descrição da empresa onde o trabalho decorreu.}

\section{Motivação e Objetivos} \label{sec:goals}

Apresenta a motivação e enumera os objetivos do trabalho terminando
com um resumo das metodologias para a prossecução dos objetivos.


\section{Projeto} \label{sec:proj}

Na continuação da secção anterior, e apenas no caso de ser um Projeto
e não uma Dissertação, esta secção apresenta resumidamente o projeto.


\section{Estrutura da Dissertação} \label{sec:struct}

Para além da introdução, esta dissertação contém mais x capítulos.
No capítulo~\ref{chap:sota}, é descrito o estado da arte e são
apresentados trabalhos relacionados.
%\todoline{Complete the document structure.}
No capítulo~\ref{chap:chap3}, é explicado em detalhe em que consiste o projeto
No capítulo~\ref{chap:chap4} praesent sit amet sem. 
No capítulo~\ref{chap:concl}  posuere, ante non tristique
consectetuer, dui elit scelerisque augue, eu vehicula nibh nisi ac
est. 
