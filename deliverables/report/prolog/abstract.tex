%!TEX root = ../mieic.tex

\chapter*{Abstract}

Not so long ago, before the Internet boom, listening or discovering new music was a challenge on its own.
Now, with a few clicks one can have on their hands such a vast music catalogue that a human mind cannot compute it.

There are dozens of online services that offer  exactly that.
Some focus on creation/generation of playlists, others try to expand their music catalogue even further, but others focus more on personalized music recommendation.
And these ones present their results to the user with a list or a grid of music artists, for example.

However, lists or grids do not give the user enough information about the relation between the results.
One could even say that they are not related to each other, which is not true.

The relations exist and can be represented as a network of interconnected artists in a graph, where a node is a music artist, and each edge between them represents a strong connection.
This is the concept that RAMA (Relational Artist MAps), a project developed at INESC Porto, uses.

From a single search, RAMA is able to draw a graph that helps the user to explore new music that might caught his/her interest in a much more natural way.

Nonetheless, when a user wants to listen to an artist's music, Youtube's stream is used.
Although one can find a large catalogue of music in Youtube, this service is not Music Oriented and the sound quality is not adequate for a music streaming service.

Youtube's stream needs to be replaced, and Spotify can provide a quality stream and an accurate music catalogue.

But how can RAMA and Spotify be integrated?

This thesis proposes a Spotify App.
Will a Spotify user experience a more pleasant and natural way of music discovery from this graphical representation of artist relations within Spotify, than its standard discovery more (with grids)?

That is the main question that this dissertation urges to answer.

\chapter*{Resumo}

Bem longe vão os tempos, antes da Internet, em que ouvir e descobrir música nova era um desafio por si só.
Agora, com alguns cliques, temos acesso a um catálogo de música tão grande, que o nosso cérebro não consegue processar.

Existem dezenas de serviços online que oferecem isso mesmo.
Alguns especializam-se na criação/geração de playlists (que funcionam como rádios), outros em expandir o catálogo de música e outros focam-se mais na sugestão e recomendação de artistas/álbuns/músicas personalizada para os utilizadores.
Estes últimos, apresentam as sugestões de conteúdo ao utilizador de uma forma rudimentar como listas ou em grelha.

No entanto, listas ou grelhas não fornecem ao utilizador qualquer tipo de informação adicional sobre a relação entre os artistas nem justificam a sua semelhança.
Até fazem parecer que não existe nenhuma relação/ligação entre os artistas recomendados, o que não é verdade.

Essas relações existem e podem ser representadas como uma rede de artistas interligados num grafo, onde cada nó é um artista de música, e cada ligação entre nós representa uma ligação forte de parecença entre os artistas.
Este é o conceito que o RAMA (Relational Artist MAps), projeto desenvolvido no INESC Porto, usa.

A partir de uma pesquisa de um artista de música, o RAMA cria e desenha um grafo que ajuda o utilizador a explorar música que lhe possa interessar de uma forma muito mais natural e informativa.

No entanto, quando um utilizador pretende ouvir uma música de um artista, é usado \emph{stream} do Youtube. 
Apesar de este oferecer um catálogo alargado de música, o mesmo não é indicado para esta funcionalidade pois não fornece uma API nativamente orientada a música, nem a qualidade de som do \emph{stream} é adequada.

A experiência musical do utilizador do RAMA poderá melhorar consideravelmente ao colmatar esta falha.
Existe por isso uma necessidade de substituir o Youtube por outro serviço mais orientado a \emph{streaming} de música de qualidade.
O Spotify é um deles. Fornece API orientada a música, e o \emph{streaming} é de qualidade adequada para este tipo de funcionalidade.

De que formas é que se pode integrar o RAMA e o Spotify?

A escolha final foi desenvolver uma aplicação (como \emph{plugin}) para o Spotify.
Será que um utilizador Spotify ao descobrir música nova de uma forma mais gráfica terá uma experiência de utilizador mais rica e natural do que o modo de descoberta \emph{standard} do Spotify (em grelha)?

Esse é o objetivo primordial desta dissertação: Tentar descobrir se utilizadores Spotify terão uma experiência melhorada ao usar a Aplicação Spotify proposta.
