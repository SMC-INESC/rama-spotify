%!TEX root = ../report.tex

\chapter{Conclusions}
\label{chap:chap5}

\section*{}


\section{Summary} % (fold)
\label{sec:summary}

  The proposed thesis focused on delivering an enhanced user experience when discovering new music in Spotify's Environment.

  By using RAMA's concept applied in the developed prototype, the user experience when discovering new music as been greatly increased.
  The users felt that RAMA's Spotify Application was natural and intuitive.

  \paragraph*{State of the Art} \hfill \\
  \indent The amount of services that use visual tools for recommending music to users are not that many, although the ones shown here are not representative of the whole spectrum.

  \paragraph*{Contextualization} \hfill \\
  \indent Given the overview of the possibilities, creating a Spotify Application to apply RAMA's concept proved to be the best option to take.

  \paragraph*{Implementation and Validation} \hfill \\
  \indent All of the proposed features were implemented within schedule. \\
  The developed prototype proved to work after the beta-testing results.
  Although, there are a lot of improvements to do, the final result was very appealing to the users.
  All of the beta-testers liked the visual experience and the majority responded positively about using the application in a regular basis to discover new music. \\

  All of the developed material (code, documentation, screenshots, demos) can be found in the project's code repository: \url{http://github.com/carsy/rama-spotify}.

% section summary (end)

\section{Discussion} % (fold)
\label{sec:discussion}


  % results discussion

% section discussion (end)

\section{Future Work} % (fold)
\label{sec:future_work}



% more tests (and more segmented)
% use more external databases to improve user experience
% all of the other features from the feedback form
% and the smc group

% section future_work (end)