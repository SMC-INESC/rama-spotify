%!TEX root = ../report.tex

\chapter{Introduction} \label{chap:intro}


\section*{}

\section{Context} \label{sec:context}


  Not so long ago, before the Internet boom, listening or discovering new music was a challenge on its own.
  Now, with a few clicks users can have on their hands such a vast music catalogue that a human mind is not able to compute.

  There is an uncountable number of music streaming services that offer exactly that\footnote{Although some of them require the users to subscribe to a monthly fee, for example, in order to fully use the service, or remove the advertisements.}.
  These services are, mostly, web based, although some offer desktop applications.
  They allow the users to play music, save their collection, create playlists and much more.
  Most of these services also have social components that allow the users to share what they're listening to with their friends, as well as playlists.

  There is always something that makes a music streaming service different from the others.
  Some services focus on creation and/or generation of playlists (8tracks~\cite{8tracks}), others try to expand their music catalogue even further (Spotify~\cite{spotify}, Rdio~\cite{rdio}), while others focus more on personalized music recommendations (Pandora~\cite{pandora}).
  The latter ones, present their music recommendations to the user with a list or a grid of music artists, for example.
  However, lists do not provide the user enough information about the relation between the results \cite{Lamere2008}.

  A possible solution to this problem is to represent the artist's similarities as a network of interconnected artists in a graph, where a node is a music artist, and each edge between them represents a connection.
  This is the concept that RAMA (Relational Artist MAps)\footnote{RAMA: \url{http://rama.inescporto.pt}}, a project developed at INESC Porto, uses \cite{Costa2008} \cite{Sarmento2009} \cite{Costa2009} \cite{Gouyon2011}.

\section{Motivation} \label{sec:motivation}

  From a single search, the original RAMA system draws a graph that helps the user to explore new music that might caught his/her interest in a much more natural way.
  Nonetheless, when a user wants to listen to an artist's music, Youtube's stream is used.
  Although one can find a large catalogue of music in Youtube, this service is not music oriented and the sound quality is not adequate for a music streaming service.
  Youtube's stream needs to be replaced.
  From the available services that provide a vast music catalogue, Spotify\footnote{\url{http://spotify.com}} provides a good quality stream and a good developer support for creating Spotify powered Applications.

  But how can RAMA and Spotify be integrated? \\

\section{Goals}
\label{sec:goals}

  There are several possibilities that Spotify has made available for developers\footnote{\url{https://developer.spotify.com}} that can help to improve RAMA's concept.
  From websites, mobile applications, native applications and even plugins for the Spotify Desktop Client, Spotify's API is very complete.
  Given the existence of some restrictions when using some APIs\footnote{for example, LibspotifySDK (\url{https://developer.spotify.com/technologies/libspotify/}) requires the developer and the user of the application to have a premium account.}, there are several aspects to take into account when choosing which API to use.

  The initial proposal was to develop a software module that implements, at least, one of the following features:

  \begin{enumerate}
    \item \label{intro:obj1} Integrate Spotify's music stream into RAMA's website
    \item \label{intro:obj2} Integrate information from the Spotify user into RAMA
    \item \label{intro:obj3} Improve RAMA's features and design
    \item \label{intro:obj4} Integrate the RAMA concept into a Spotify Application
    \item \label{intro:obj5} Integrate RAMA's playlist generation into a Spotify Application
    \item \label{intro:obj6} Integrate some of the above mentioned modules into a Mobile Application
  \end{enumerate}

  All possibilities will be explored with further detail in Chapter~\ref{chap:chap3}.
  In the end, this dissertation proposes a Spotify Application\footnote{\url{https://developer.spotify.com/technologies/apps}} (module~\ref{intro:obj4}) that works like a plugin to the Spotify's Desktop Client, i.e., it should add something to Spotify.
  This is a very appealing solution: Spotify Users will have the chance to continue using Spotify as they would normally do, but with an extra help to discover new music by using RAMA's application \emph{inside} Spotify. 
  This method works on the assumption that Spotify's music discovery mode can be improved using a visual tool like RAMA.

  After specifying the requirements, a prototype will be developed.
  This approach requires a solution to the following question: Will a Spotify User experience a more pleasant and natural way of music discovery from this graphical representation of artist relations within Spotify, than its standard discovery mode?
  To answer that question and to evaluate and validate the final prototype, end-user testing will be done to compare the user experience of discovering new music with or without the developed application.

\section{Methodologies}
\label{sec:methodologies}

  The following work methods consist of four phases: initial research on the state of the art of the current applications that are similar to RAMA's approach; contextualization of the Spotify's environment given the tools available for users and developers; definition and implementation of the prototype's user requirements and validation of the developed prototyped by users.

  % SOTA
  The initial state of the art research was done to assess the main features of the music services that offer any sort of music discovery interface.
  There are an uncountable number of them, and so, focus was given to the ones that use visual tools.
  Their main features were analysed, as well their pros and cons.

  % Context
  Next, Spotify's environment was introduced from the user's perspective and the developer's perspective.
  At first, it was important to establish how a Spotify user goes about discovering new music, i.e., what tools are available.
  After that, the available tools for developers of Spotify-powered applications were analysed in detail.
  The goal for the end of this phase was to determine what type of application to be developed as the prototype. 

  % Implementation
  By this point the user requirements of the application were defined and implemented.
  The tools used in the prototype's development were important to help automate most of the repetitive tasks: source code version management, testing, package managing, building and deploying.
  All of the development processes were important to keep a maintainable development environment.
  After each feature implementation, a small constant group of alpha-testers gave feedback and suggestions to be introduced in the following version of the prototype - this was the main development cycle that continuously made the prototype evolve into the defined user requirements.

  % User Validation
  After the prototype implemented the defined user requirements, it was submitted for user evaluation.
  The beta-testers were asked to discover new music with the developed application and then fill a short questionnaire that was meant to capture their first hands-on experience with the prototype.
  The results of the experiment were analysed to interpret the success to the application in meeting its goals.

\section{Report's Structure}
\label{sec:structure}
  
  Thesis report presents 4 additional chapters:

  \begin{description}
    \item[Chapter~\ref{chap:chap2}: State of the Art] \hfill \\
      Initial research on the current state of the art.
      Detailed analysis on the services that provide a platform for users to listen and discover new music.
      Although, focus will be given to the ones that use visual tools.
    \item[Chapter~\ref{chap:chap3}: Contextualization] \hfill \\
      Detailed analysis of the Spotify environment from the users perspective (applications available, e.g.) and from the developers perspective (available APIs, e.g.) in order to give a much more insightful view when determining the feasibility of the prototype's requirements.
    \item[Chapter~\ref{chap:chap4}: Implementation and Validation] \hfill \\
      Definition and implementation of: the prototype's main features/requirements; the development processes and the user validation processes.
    \item[Chapter~\ref{chap:chap5}: Discussion and Future Work] \hfill \\
      Discussion of the results and definition of future work to be done in the prototype (improvements, features, etc).
  \end{description}
