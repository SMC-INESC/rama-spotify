%!TEX root = ../report.tex

\chapter{Introduction} \label{chap:intro}


\section*{}

\section{Context} \label{sec:context}


  Not so long ago, before the Internet boom, listening or discovering new music was a challenge on its own.
  Now, with a few clicks one can have on their hands such a vast music catalogue that a human mind is not able to compute.

  There is an uncountable number of music streaming services that offer exactly that\footnote{Although some of them require the users to subscribe to a monthly fee, for example, in order to fully use the service, or remove the advertisements.}.
  These services are, mostly, web based, although some offer desktop applications.
  They allow the users to play music, save their collection, create playlists and much more.
  Most of these services also have social components that allow the users to share what they're listening to with their friends, as well as playlists and much more.

  There is always something that makes a music streaming service different from the others.
  Some services focus on creation/generation of playlists (8tracks\footnote{\url{http://8tracks.com}}), others try to expand their music catalogue even further (Spotify\footnote{\url{http://spotify.com}}, Rdio\footnote{\url{rdio.com}}), while others focus more on personalized music recommendations (Pandora\footnote{\url{pandora.com}}).
  The latter ones, present their music recommendations to the user with a list or a grid of music artists, for example.
  However, lists do not provide the user enough information about the relation between the results \cite{Lamere2008}.
  One could even say that they are not related to each other, which is not true.
  
  The relations exist and can be represented as a network of interconnected artists in a graph, where a node is a music artist, and each edge between them represents a connection.
  This is the concept that RAMA (Relational Artist MAps), a project developed at INESC Porto, uses \cite{Costa2008} \cite{Sarmento2009} \cite{Costa2009} \cite{Gouyon2011}.

  The concept is not new, and in Chapter~\ref{chap:chap2} some services that use visual tools will be analysed in detail.


\section{Motivation and Goals} \label{sec:goals}

  From a single search, RAMA draws a graph that helps the user to explore new music that might caught his/her interest in a much more natural way.

  Nonetheless, when a user wants to listen to an artist's music, Youtube's stream is used.
  Although one can find a large catalogue of music in Youtube, this service is not music oriented and the sound quality is not adequate for a music streaming service.

  Youtube's stream needs to be replaced.
  From the available services that provide a vast music catalogue, Spotify\footnote{\url{http://spotify.com}} provides a good quality stream and a good developer support for creating Spotify powered Applications.

  But how can RAMA and Spotify be integrated? \\

  There are several possibilities that Spotify has made available for developers\footnote{\url{https://developer.spotify.com}} that can help to improve RAMA's concept.
  From websites, mobile applications, native applications and even plugins for the Spotify Desktop Client, Spotify's API is very complete.

  Given the existence of some restrictions when using some APIs\footnote{for example, LibspotifySDK (\url{https://developer.spotify.com/technologies/libspotify/}) requires the developer and the user of the application to have a premium account.}, there are several aspects to take into account when choosing which API to use. \\

  The initial proposal was to develop a software module that implements, at least, one of the following features:

  \begin{enumerate}
    \item \label{intro:obj1} Integrate Spotify's music stream into RAMA's website
    \item \label{intro:obj2} Integrate information from the Spotify user into RAMA
    \item \label{intro:obj3} Improve RAMA's features and design
    \item \label{intro:obj4} Integrate the RAMA concept into a Spotify Application
    \item \label{intro:obj5} Integrate RAMA's playlist generation into a Spotify Application
    \item \label{intro:obj6} Integrate some of the above mentioned modules into a Mobile Application
  \end{enumerate}

  All possibilities will be explored with further detail in Chapter~\ref{chap:chap3}.
  In the end, this dissertation proposes a Spotify Application\footnote{\url{https://developer.spotify.com/technologies/apps}} (module~\ref{intro:obj4}) that works like a plugin to the Spotify's Desktop Client, i.e., it should add something to Spotify's Application.
  This is a very appealing solution: Spotify Users will have the chance to continue using Spotify as they would normally do, but with an extra help to discover new music by using RAMA's application \emph{inside} Spotify. 
  This method works on the assumption that Spotify's music discovery mode can be improved using a visual tool like RAMA. \\

  After specifying the requirements, a prototype will be developed.
  This approach urges to answer the following question: Will a Spotify User experience a more pleasant and natural way of music discovery from this graphical representation of artist relations within Spotify, than its standard discovery mode?

  To answer that question and to evaluate and validate the final prototype, end-user testing will be done to compare Spotify's users experience of discovering new music with or without the developed application.


\section{Requirements}
\label{sec:requirements}

  This application is meant to be an extra mode for discovering new music in Spotify.

  This way, a visual representation of an artist network with a graph, similar to RAMA, is proposed.

  The application runs inside the Spotify environment (Spotify's Desktop Client) were its main features are: 

  \begin{itemize}
    \item visualization of relations between artists;
    \item edition of the visualization parameters (branching and depth);
    \item edition of the graph by expanding and deleting the nodes;
    \item visualize tags/genres (that describe an artist) in the graph representation.
  \end{itemize}

  These are the minimum requirements that the application must implement.


\section{Methodologies} % (fold)
\label{sec:methodologies}
  
  The proposed work phases contemplate:

  \begin{description}
    \item[State of the Art (Chapter~\ref{chap:chap2})] \hfill \\
      Initial research on the current state of the art.
    \item[Contextualization (Chapter~\ref{chap:chap3})] \hfill \\
      Detailed analysis of the Spotify environment from the users perspective (applications available, e.g.) and from the developers perspective (available APIs, e.g.) in order to give a much more insightful view when determining the feasibility of the prototype's requirements.
    \item[Implementation and Validation (Chapter~\ref{chap:chap4})] \hfill \\
      Definition and implementation of: the prototype's main features/requirements; the development processes and the user validation processes.
    \item[Future Work (Chapter~\ref{chap:chap5})] \hfill \\
      Discussion of the results and definition of future work to be done in the prototype (improvements, features, bugs, etc).
  \end{description}

% section methodologies (end)

\section{Dissertation Structure} \label{sec:struct}

  This dissertation contains four additional chapters.

  In chapter \ref{chap:chap2}, related works will be presented to evaluate the current state of the art.

  In chapter \ref{chap:chap3}, the project's details will be explained, starting with an introduction to the Spotify Apps' development environment and the role of the technologies used during the development of the prototype.

  In chapter \ref{chap:chap4}, a more detailed explanation about the developed prototype will be given, as well as some challenges and problems encountered during the development process.

  Chapter \ref{chap:chap5} concludes this report.
