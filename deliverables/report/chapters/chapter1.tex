%!TEX root = ../mieic.tex

\chapter{Introduction} \label{chap:intro}


\section*{}

\section{Context} \label{sec:context}


Not so long ago, before the Internet boom, listening or discovering new music was a challenge on its own.
Now, with a few clicks one can have on their hands such a vast music catalogue that a human mind is not able to compute.

There is an uncountable number of online services that offer exactly that.
Some focus on creation/generation of playlists, others try to expand their music catalogue even further, while others focus more on personalized music recommendations.
Most of these, present their results to the user with a list or a grid of music artists, for example.

However, lists or grids do not give the user enough information about the relation between the results \cite{Lamere2008}.
One could even say that they are not related to each other, which is not true.

The relations exist and can be represented as a network of interconnected artists in a graph, where a node is a music artist, and each edge between them represents a strong connection.
This is the concept that RAMA (Relational Artist MAps), a project developed at INESC Porto, uses. \cite{Costa2008} \cite{Sarmento2009} \cite{Costa2009} \cite{Gouyon2011}

\section{Motivation and Goals} \label{sec:goals}


From a single search, RAMA is able to draw a graph that helps the user to explore new music that might caught his/her interest in a much more natural way.

Nonetheless, when a user wants to listen to an artist's music, Youtube's stream is used.
Although one can find a large catalogue of music in Youtube, this service is not music oriented and the sound quality is not adequate for a music streaming service.

Youtube's stream needs to be replaced, and Spotify\footnote{http://spotify.com} can provide a quality stream and an accurate music catalogue.

But how can RAMA and Spotify be integrated? \\

Several possibilities were analysed.

\begin{description}
  \item[Spotify Play Button\footnote{https://developer.spotify.com/technologies/widgets/spotify-play-button/}] \hfill \\
    A Spotify widget that can be embedded in RAMA. 
  \item[Integrate Spotify User's profile data in RAMA] \hfill \\
    To help complement artist recommendations.
  \item[Spotify App\footnote{https://developer.spotify.com/technologies/apps/}] \hfill \\
    A plugin to Spotify's desktop client
\end{description}


This dissertation proposes a Spotify App.
Will a Spotify user experience a more pleasant and natural way of music discovery from this graphical representation of artist relations within Spotify, than its standard discovery mode (with lists)?

That is the main question that this dissertation urges to answer.

Moreover, to evaluate and validate the final product, end-user testing will be done to compare Spotify's user experience with and without the developed application. 


\section{Project} \label{sec:proj}

The app is meant to be an extra mode for discovering new music in Spotify.

This way, a visual representation of an artist network with a graph, similar to RAMA, is proposed.

The application runs inside the Spotify environment (Spotify's Desktop Client) where its main features are: visualization of relations between artists, starting by the current playing artist; ability to grow the graph by expanding nodes; visualize tags (that describe an artist) in the graph representation.

The tools used in the development of the application were:

\begin{description}
  \item[Spotify Desktop Client] \hfill \\
    The developed application is integrated in the Spotify's desktop client.
  \item[Webkit Development Tools - webkit.org] \hfill \\
    This is the engine used to run Spotify Applications.
  \item[Npmjs - npmjs.org] \hfill \\
    Package manager for development dependencies.
  \item[Bower - bower.io] \hfill \\
    Package manager for runtime dependencies.
  \item[Gruntjs - gruntjs.com] \hfill \\
    Manager for automating tasks. Very usefull for tests, code optimization and other repetitive tasks.
  \item[Vis.js - visjs.org] \hfill \\
    Visualization framework.
\end{description}


\section{Dissertation Structure} \label{sec:struct}

This dissertation contains four additional chapters.

In chapter \ref{chap:chap2}, related works will be presented to evaluate the current state of the art.

In chapter \ref{chap:chap3}, the project's details will be explained, starting with an introduction to the Spotify Apps' development environment and the role of the technologies used during the development of the prototype.

In chapter \ref{chap:chap4}, a more detailed explanation about the developed prototype

Chapter \ref{chap:concl} concludes this report.
